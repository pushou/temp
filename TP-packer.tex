% !TEX root = TP-packer.tex
\input{../inc/encommun.tex}

\usepackage{titlepic}
\title{Construisez vos briques de bases Cloud avec "Packer"}
\titlepic{\includegraphics[width=0.2\textwidth]{logo-cloudlab-red.png}}

\date{octobre 2024}

% pour afficher (ou pas) les réponses dans le document
%\printanswers
\noprintanswers

\usepackage{graphicx}
\graphicspath{{./images/}}


\author{Jean-Marc Pouchoulon}
%\author{\IEEEauthorblockN{Jean-Marc Pouchoulon}
%\IEEEauthorblockA{Département Réseaux et Télécoms\\
%IUT Béziers\\
%Université de Montpellier\\
%Email: jean-marc.pouchoulon@iutbeziers.fr}}

% pour afficher ( ou pas) les réponses dans le document
%\printanswers


\begin{document}
\maketitle
\section{Avant de commencer}

\subsection{Pré-requis, recommandations et notation du TP.}

Vous devez avoir Docker, VirtualBox et KVM/libvirt installés sur votre machine afin de réaliser ce TP.
Le source est bâti pour fonctionner avec VirtualBox 7.

Vous pouvez utiliser la version 6 de VirtualBox mais il faudra modifier
le fichier de configuration en retirant 
["modifyvm", "{{ .Name }}", "--nat-localhostreachable1", "on"]]
de la ligne commençant par "vboxmanage..."


\section{Installation de Packer}

Installez Packer en suivant \href{https://developer.hashicorp.com/packer/tutorials/docker-get-started/get-started-install-cli}{la procédure officielle}

Installez les plugins suivants:
\begin{bashcode}
packer plugins install github.com/hashicorp/virtualbox
packer plugins install github.com/hashicorp/qemu
packer plugins install github.com/hashicorp/docker
\end{bashcode}

\section{Réalisez votre premier build d'une image Docker avec Packer}

Buildez votre premier container avec cet exemple de build sur le site officiel de Packer : \url{https://www.packer.io/intro/getting-started/build-image.html}

\section{Premier build d'une VM avec Packer}
Récupérez votre VM Debian avec cette \href{https://github.com/pushou/AMORCE-TP-PACKER-R5-devcloud-11.git}{première configuration}.
et buildez votre VM avec Packer. (Vous serez amenés à mettre à jour le checksum et la version de Debian)
Répondez ensuite aux questions suivantes relation au fichier .hcl:
\begin{questions}
    \question Quel est le bloc essentiel de cette configuration ?
    \question A quoi sert le block plugin ?
    \question Que lance le block build ?
    \question A quoi sert la variable "checksum" ?
    \question Que lance la "bootcommand" ? 
    \question A quoi sert le fichier preseed.cfg ?
    \question Lors du build que lance packer afin de se servir du fichier ?
    \begin{solution}
        \begin{itemize}
            \item \textbf{bloc build} : C’est le bloc essentiel qui indique quelles sont les actions à mener pour construire l’image machine 
            finale. Dans ce bloc, vous avez la possibilité de renseigner des providers (source) et les provisonners. 
            \item \textbf{bloc source} : Ce bloc permet de configurer l’environnement cible où sera construite l’instance. 
            \item \textbf{bloc variables}: définit les variables utilisées dans le template
            \item \textbf{bloc provisioner} : permet de configurer les actions à mener sur l’instance avant de la packeriser.
        \end{itemize}
    \end{solution}
    \question Quel est le provisionner utilisé ?   
    \question Quel est le format de la VM générée ?
    \question Que font chacun des scripts shell utilisés par Packer ?
\end{questions}

\section{Amélioration de la configuration}

\begin{questions}
    \question Ajoutez un serveur apache2 via un provisionner shell "inline" dans le fichier de configuration et rebuildez la VM.
    \question Désactivez IPV6 lors du build de la VM.
    \question Lors du "build" rajoutez dans le fichier /etc/hosts de la VM la ligne suivante:
    10.255.255.135  registry.iutbeziers.fr
    \question Ajoutez un wordpress \footnote{voir \url{https://galaxy.ansible.com/Oefenweb/wordpress}} via Ansible dans le build de votre VM ?
    \question Utilisez cette source pour construire une image KVM/libvirt.

    \begin{bashcode}
source "qemu" "qemu" {
    boot_command     = ["<esc><wait>", "auto ", "net.ifnames=0 ",   
                       "preseed/url=http://{{ .HTTPIP }}:{{ .HTTPPort }}/debian-12/preseed.cfg ", "<enter>"]
    boot_wait        = "15s"
    disk_size        = "${var.disk_size}"
    headless         = "${var.headless}"
    http_directory   = "http"
    iso_checksum     = "${var.iso_checksum_type}:${var.iso_checksum}"
    iso_url          = "https://cdimage.debian.org/debian-cd/current/amd64/iso-cd/debian-12.1.0-amd64-netinst.iso"
    output_directory = "output-debian-12-amd64-small-qemu"
    qemuargs         = [["-m", "${var.memory}"], ["-smp", "${var.cpus}"]]
    shutdown_command = "sudo systemctl poweroff"
    ssh_password     = "vagrant"
    ssh_timeout      = "${var.ssh_timeout}"
    ssh_username     = "vagrant"
    vm_name          = "packer-debian-12-amd64-kvm-small"
}
    \end{bashcode}
    \begin{solution}
        \begin{bashcode}
sudo bash -c "cat <<_EOF >> /etc/hosts
10.255.255.135  registry.iutbeziers.fr
_EOF
"
sudo sysctl -w net.ipv6.conf.all.disable_ipv6=1
sudo sysctl -w net.ipv6.conf.default.disable_ipv6=1

sudo bash -c "cat <<_EOF >> /etc/sysctl.conf
net.ipv6.conf.all.disable_ipv6=1
net.ipv6.conf.default.disable_ipv6=1
_EOF
"
sudo bash -c "cat <<_EOF >> /etc/hosts
registry.iutbeziers.fr 10.255.255.135
_EOF
"
\end{bashcode}
\end{solution}
\end{questions}



\end{document}
