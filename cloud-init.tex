% !TEX root = cloud-init.tex
\input{../inc/encommun.tex}

\usepackage{titlepic}
\title{Cloud-init}


\date{octobre 2025}

% pour afficher (ou pas) les réponses dans le document
%\printanswers
\noprintanswers

\usepackage{graphicx}
\graphicspath{{./images/}}



%\author{Jean-Marc Pouchoulon}

% pour afficher ( ou pas) les réponses dans le document
%\printanswers


\begin{document}
\maketitle


Cloud-init est un ensemble d'outils permettant de personnaliser une instance cloud lors de son premier démarrage.
Il est utilisable dans la majorité des solutions de virtualisation et cloud.
Incus est un fork de LXD permettant de gérer des containers systèmes, des machines virtuelles K-VM et des containers applicatifs.
Il est possible d'utiliser cloud-init avec Incus c'est ce que nous allons faire dans ce TP 
dont l'objectif est de construire une K-VM debian pour en faire une station de travail "Ansible ready".
La notation est individuelle, le bon fonctionnement de votre K-VM sera vérifié par un enseignant. 
Le fichier profile sera évalué et est à rendre sur github education (sur Moodle)
La rédaction d'un rapport n'est pas demandée.

Vous pouvez vous aidez de \href{https://linuxcontainers.org/incus/docs/main/cloud-init/}{la documentation de Incus} 
et de la documentation de \href{https://cloudinit.readthedocs.io/en/latest/topics/examples.html}{cloud-init}. 


\section{Installations}


\subsection {Installez Incus sur votre poste.}

Vous devez avoir qemu-kvm installé sur votre machine afin de réaliser ce TP.

Vérifiez avec virt-host-validate que votre poste est prêt pour faire de la virtualisation.

Utilisez le repos de \url{https://pkgs.zabbly.com}. pour installer Incus.


\subsection{Créer un profile Incus appelé ansible en copiant le profile par defau.}

C'est ce profile que vous allez utiliser pour créer notre K-VM ansiblevm.

\begin{bashcode}
incus profile copy default ansible
\end{bashcode}

Lancez la VM avec le profile ansible et une image debian.
\begin{bashcode}
incus launch images:debian/13/cloud ansiblevm  --profile ansible
\end{bashcode}

\section{Cahier des charges actions à faire faire à cloud-init}

\begin{itemize}
\item L'utilisateur ansible doit être créé et accessible en ssh à partir de votre clef publique initialisé  cloud-init.  
Il doit être aussi dans le groupe sudo et pouvoir faire un sudo soot sans mot de passe.
\item Votre VM doit avoir comme hostname  ansiblevm.
\item Votre VM doit aller chercher vos playbooks sur un repos git à intervalle régulier.
\item L'adresse ip de la vm doit être fixée par cloud-init ainsi que la passtemerelle et le fns de l'IUT.
\item Ansible doit être installé et fonctionnel.
\end{itemize}

\end{document}