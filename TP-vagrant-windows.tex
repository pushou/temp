% !TEX root = TP-vagrant-windows.tex
\input{../inc/encommun.tex}

\usepackage{titlepic}
\title{Préparation à la SAE: Vagrant et administration Windows}
\titlepic{\includegraphics[width=0.2\textwidth]{logo-cloudlab-red.png}}

\date{octobre 2023}

% pour afficher (ou pas) les réponses dans le document
%\printanswers
\noprintanswers

\usepackage{graphicx}
\graphicspath{{./images/}}


\author{Jean-Marc Pouchoulon}
%\author{\IEEEauthorblockN{Jean-Marc Pouchoulon}
%\IEEEauthorblockA{Département Réseaux et Télécoms\\
%IUT Béziers\\
%Université de Montpellier\\
%Email: jean-marc.pouchoulon@iutbeziers.fr}}

% pour afficher ( ou pas) les réponses dans le document
%\printanswers


\begin{document}
\maketitle

\section{Installation d'un couple de machine Windows avec Vagrant}

Installez zsh et travaillez à partir de ce shell.
L'utilisateur et le mot de passe des VM est vagrant/vagrant.

Vérifiez que libvirt/qemu sont installés sur votre machine:
\begin{bashcode}
sudo apt -y install qemu-system libvirt-daemon-system freerdp2-x11 zsh
\end{bashcode}

Installez aussi les plugins suivants:
\begin{bashcode}
vagrant plugin install vagrant-libvirt
vagrant plugin install vagrant-scp 
# vérifiez
vagrant plugin list
# positionnez les variables suivantes 
export VAGRANT_DEFAULT_PROVIDER=libvirt 
export OPENSSL_CONF=./openssl.cnf
\end{bashcode}

Clonez le projet suivant et démarrer la maquette : 
\begin{bashcode}
zsh
git clone https://github.com/pushou/but-503-vagrant.git
\end{bashcode}

Les VM sont natés, il vous faut créer un pont réseau.
\begin{bashcode}
virsh -c qemu:///system net-create ./bridge-vagrant.xml
\end{bashcode}


listez les adresses réseaux des deux VM avec WinRM et Vagrant:
\begin{bashcode}
vagrant winrm -s cmd -c ipconfig win-1
vagrant winrm -s cmd -c ipconfig win-2

    Windows IP Configuration
    
    
    Ethernet adapter Ethernet Instance 0:
    
       Connection-specific DNS Suffix  . :
       Link-local IPv6 Address . . . . . : fe80::951f:88fe:d098:52e9%6
       IPv4 Address. . . . . . . . . . . : 192.168.121.123
       Subnet Mask . . . . . . . . . . . : 255.255.255.0
       Default Gateway . . . . . . . . . : 192.168.121.1
...
\end{bashcode}

Lancez le script run.sh pour démarrer la maquette.

\subsection{Tips \& Tricks Vagrant}

\begin{bashcode}
# instanciez ou démarrez les VM
vagrant up 
# listez les VM
vagrant status    
# arrêtez une VM
vagrant halt win-1
# détruisez une VM 
vagrant destroy win-1
# se connecter en ssh sur une VM
vagrant ssh win-1
# lancez l'interface graphique 
xfreerdp  /u:vagrant /p:vagrant /v:192.168.121.123:3389  /cert-ignore  /dynamic-resolution
\end{bashcode}


Modifiez la "time zone" de la VM en lançant timedate.cpl  \textbf{en tant qu'administrateur}.

\section{Installation d'un "Domain Controller" sur win-1 et rattachement de win-2 au domaine}

Créez un domaine \textit{VOSINITIALES.local} en suivant ce 
\href{https://github.com/pushou/pushou_public_pdf/blob/main/BUT/R5.cyber.11/installation_ad_en_images.pdf}{\emph{document}}.
Rattachez ensuite la VM win-2 au domaine en suivant \href{https://github.com/pushou/pushou_public_pdf/blob/main/BUT/R5.cyber.11/join_domain_en_images.pdf}
{cet \emph{autre document}}

%\section{Automatiser la mise au domaine}(pour les champions uniquement)
faite valider par l'enseignant votre mise au domaine.

\end{document}
