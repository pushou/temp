% !TEX root = TP-terraform.tex
\input{../inc/encommun.tex}

\usepackage{titlepic}
\title{Terraform \& Pulumi}
\titlepic{\includegraphics[width=0.2\textwidth]{logo-cloudlab-red.png}}

\date{octobre 2024}

% pour afficher (ou pas) les réponses dans le document
%\printanswers
\noprintanswers

\usepackage{graphicx}
\graphicspath{{./images/}}



\author{Jean-Marc Pouchoulon}

% pour afficher ( ou pas) les réponses dans le document
%\printanswers


\begin{document}
\maketitle
\section{Avant de commencer}

\subsection{Pré-requis, recommandations et notation du TP.}

Vous devez avoir Docker, KVM/libvirt installés sur votre machine afin de réaliser ce TP.
Pour certaines parties une serveur proxmox sera nécessaire.
Terrafom doit être installé sur votre machine. (voir \href
{https://developer.hashicorp.com/terraform/tutorials/aws-get-started/install-cli}{la procédure officielle})


Installez Terraform en suivant \href{https://developer.hashicorp.com/packer/tutorials/docker-get-started/get-started-install-cli}{la procédure officielle}

Vous ferez valider votre travail par l'enseignant à la fin de chaque partie.

\section{Terraform \& Docker}

\subsection{Commandes de bases Terraform}
\begin{questions}
    
\question A l'aide de ces deux fichiers providers.tf main.tf générez un conteneur docker avec Terraform. 

\begin{terraformcode}
    # fichier providers.tf

    terraform {
        required_providers {
          docker = {
            source  = "kreuzwerker/docker"
            version = "3.0.2"
          }
        }
      }

    provider "docker" {
        host = "unix:///var/run/docker.sock"
    }

\end{terraformcode}

\begin{terraformcode} 
      # fichier main.tf    
      # Pulls the image
      resource "docker_image" "ubuntu" {
        name = "ubuntu:latest"
      }
      
      # Create a container
      resource "docker_container" "foo" {
        image = docker_image.ubuntu.image_id
        name  = "foo"
      }
\end{terraformcode}

Utilisez les commandes suivantes:
\begin{bashcode}
  terraform init 
  terraform validate
\end{bashcode}

\question Que contient la directory .terraform ? le fichier .terraform.lock.hcl (voir \url{https://developer.hashicorp.com/terraform/tutorials/cli/init} pour vous aidez)

\question Rajoutez le provider suivante et refaite un validate. que se passe-t-il ?

\begin{terraformcode}
random = {
  source  = "hashicorp/random"
  version = "3.5.1"
}
\end{terraformcode}
\question Générez un plan d'exécution et examinez-le avec les commandes suivantes:

\begin{bashcode}
terraform plan -out "tfplan"
terraform show "tfplan"
\end{bashcode}

\question Appliquez le plan d'exécution avec la commande suivante:

\begin{bashcode}
terraform  apply -auto-approve "tfplan"
\end{bashcode}

\question Utilisez l'instruction count=3 afin d'obtenir 3 conteneurs docker et modifiez le code en utilisant la variable \$\{count.index\} .

\question 
\end{questions}

\subsection{Un peu plus loin avec Docker et Terraform}
\begin{questions}

\question Modifiez ce code afin d'obtenir une instance de container en état running en utilisant les instructions suivantes :
\begin{terraformcode}
must_run = true
publish_all_ports = true 
command = [.... 
\end{terraformcode}
\begin{solution}
\begin{terraformcode}

    # Pulls the image
    resource "docker_image" "ubuntu" {
      name = "ubuntu:latest"
    }
    
    # Create a container
    resource "docker_container" "foo" {
      image = docker_image.ubuntu.image_id
      name  = "foo"
      must_run          = true
      publish_all_ports = true
    
      command = [
        "tail",
        "-f",
        "/dev/null"
      ]
    }
\end{terraformcode}


\end{solution}


\question Modifiez ce code afin d'obtenir une instance running de registry.iutbeziers.fr/debianiut:latest 

Vous devez avoir l'image sur votre machine avant de lancer le code terraform.

Vous devez référencer l'image de cette façon:

\begin{terraformcode}
data "docker_image" "debianiut" {
    name = "registry.iutbeziers.fr/debianiut:latest"
  }      
# image registry.iutbeziers.fr/debianiut:latest
  resource "docker_image" "debianiut" {
  name = data.docker_image.debianiut.name
}
\end{terraformcode}
et l'utilisez ensuite dans la ressource docker\_container.

\begin{solution}
\begin{terraformcode}
    data "docker_image" "debianiut" {
        name = "registry.iutbeziers.fr/debianiut:latest"
      }
      
      # Pulls the image
      resource "docker_image" "debianiut" {
        name = data.docker_image.debianiut.name
      }
      
      # Create a container
      resource "docker_container" "debianiut" {
        image = docker_image.debianiut.name
        name  = "debianiut"
        must_run          = true
        publish_all_ports = true
      
        command = [
          "tail",
          "-f",
          "/dev/null"
        ]
      }
\end{terraformcode}
\end{solution}

\question Supprimez votre environnement containérisés maintenant.

\begin{bashcode}
terraform destroy -auto-approve 
\end{bashcode}

\question  Instanciez des containers sur un VM distante.

Tout d'abord en utilisant SSH.
 
\begin{terraformcode}
provider "docker" {
        host     = "ssh://user@remote-host:22"
        ssh_opts = ["-o", "StrictHostKeyChecking=no", "-o", "UserKnownHostsFile=/dev/null"]
}
\end{terraformcode}


Puis rendez accessible Docker en remote sur le port tcp 2375 via une socket TCP en éditant le service Docker:

\begin{bashcode}
systemctl edit docker.service    
\end{bashcode}
Rajouter les lignes suivantes:
\begin{inicode}
[Service]
ExecStart=
ExecStart=/usr/bin/dockerd -H fd:// -H tcp://0.0.0.0:2375
\end{inicode}
Redémarrez le service Docker.


Modifiez votre providers.tf afin d'utiliser le port tcp 2375:
\begin{terraformcode}
    provider "docker" {
      host = "tcp://your-host-ip:2375/"
    }
\end{terraformcode}
nb: cette configuration ouverte à tout n'est pas sécurisée. Il aurait fallu 
utiliser TLS et des certificats clients.

\end{questions}



\subsection{Terraform wordpress} 

Utilisez \href{https://connect.ed-diamond.com/GNU-Linux-Magazine/glmf-240/utilisez-terraform-pour-vos-projets-docker}{l'article} sous "Creative Common" de Julien Morot pour créer un wordpress avec Terraform.


\section{Terraform et KVM/libvirt}

\subsection{pré-requis}
Tout d'abord vous devez éditer le fichier /etc/libvirt/qemu.conf et remplir la valeur suivante:
\begin{textcode}
security_driver = "none"
\end{textcode}
Puis redémarrer le service libvirt:
\begin{bashcode}
systemctl restart libvirtd
\end{bashcode}

On va utiliser le \href{https://registry.terraform.io/providers/dmacvicar/libvirt/latest}{provider libvirt} pour terraform.
Du à un bug il vous faut utiliser la version 1.5.7 de terraform. (voir \url{https://releases.hashicorp.com/terraform/1.5.7/})

\begin{terraformcode}
terraform {
  required_providers {
    libvirt = {
      source = "dmacvicar/libvirt"
    }
  }
  # instance the provider
  provider "libvirt" {
    uri = "qemu:///system"
  }
}

\end{terraformcode}

\subsection{Créez une machine virtuelle ubuntu KVM via Terraform}

Vous pouvez vous inspirer du \href{https://blog.stephane-robert.info/post/terraform-provider-libvirt} {code de Stéphane Robert}.
La configuration de la machine virtuelle utilisera cloud-init pour "customiser"  la machine virtuelle.


\subsection{Créez une machine virtuelle Ubuntu et Rocky via Terraform sur Kvm}

Récupérez le projet :
\begin{bashcode}
git clone https://github.com/pushou/terraform-student.git  
\end{bashcode}

Récupérer les images cloud suivantes:

\begin{bashcode}
wget https://dl.rockylinux.org/pub/rocky/9/images/x86_64/Rocky-9-GenericCloud.latest.x86_64.qcow2
wget https://cloud-images.ubuntu.com/jammy/current/jammy-server-cloudimg-amd64.img
\end{bashcode}

Re-dimensionnez les images et changer les mots de passes avec les commandes suivantes:
\begin{bashcode}
qemu-img info jammy-server-cloudimg-amd64.img
qemu-img resize jammy-server-cloudimg-amd64.img +20G
qemu-img info Rocky-9-GenericCloud.latest.x86_64.qcow2
qemu-img resize  Rocky-9-GenericCloud.latest.x86_64.qcow2 +20G 
mv Rocky-9-GenericCloud.latest.x86_64.qcow2 rocky.qcow2
mv jammy-server-cloudimg-amd64.img ubuntu.qcow2 
export LIBGUESTFS_BACKEND=direct
sudo virt-customize -a ubuntu.qcow2 --root-password password:root
sudo virt-customize -a rocky.qcow2 --root-password password:root
sudo virt-customize -a ubuntu.qcow2 --timezone "Europe/Paris"
sudo virt-customize -a rocky.qcow2 --timezone "Europe/Paris"
\end{bashcode}

Rajoutez à ce build une machine virtuelle Debian stable.

\section{Création d'un projet Terraform ou Pulumi sur un hyperviseur PROXMOX (>=8)}
Vous travaillerez en binôme sur cette partie.
Vous devez avoir un proxmox 8 installé sur un serveur de la salle:

Suivez au choix un des ces deux articles de Julien Briault et Stéphane Robert:
\begin{itemize}

  %\item \href{https://blog.jbriault.fr/2021/03/26/terraform-proxmox/}{Terraform et Proxmox}
  \item \href{https://blog.stephane-robert.info/docs/virtualiser/type1/proxmox/terraform/}{Terraform et Proxmox}
  \item \href{https://blog.jbriault.fr/pulumi-proxmox-cloudinit/}{Pulumi et Proxmox}
\end{itemize}
 
Attention il vous faudra donner plus de droits qu'indiqués dans les articles au rôle Terraform ou Pulumi 
pour pouvoir créer des VM.

\begin{bashcode}
  pveum role add Terraform -privs "VM.Allocate VM.Clone VM.Config.CDROM VM.Config.CPU \
  VM.Config.Cloudinit VM.Config.Disk VM.Config.HWType VM.Config.Memory VM.Config.Network \
  VM.Config.Options VM.Monitor VM.Audit VM.PowerMgmt Datastore.AllocateSpace Datastore.Audit SDN.Use SDN.Audit"
\end{bashcode}

\end{document}